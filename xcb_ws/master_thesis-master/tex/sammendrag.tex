%!TEX root = ../Thesis.tex
\chapter{Sammendrag}\label{cha:sammendrag}
Ubemannede multirotor luftfartøys (UML) gode evne til både å manøvrere, og til å kunne stå stille i luften, gjør dem til svært populære plattformer med mange ulike bruksområder. Deres ulempe er begrensende flytid, hvilket gjør UML uegnet til mange fullautonome operasjoner. Ved å introdusere andre farkoster for distribusjon og innhenting, automatisk batteribytte og vedlikehold av UML, kan bruken av multirotor luftfartøy i autonome operasjoner nå langt utover dagens bruksområder. For å kunne oppnå en slik samhandling mellom UML og farkosten det skal samarbeide med, må imidlertid multirotor luftfarøtyet kunne lande autonomt på farkosten mens farkosten er i bevegelse. 

Denne avhandlingen omhandler temaet autonom landing med UML på en farkost i bevegelse. For å kunne oppnå dette har det blitt brukt tradisjonelle navigasjonssensorer i kombinasjon med kamerabasert målesystem. Avlesninger fra det kamerabaserte målesystemet og navigasjonssensorene er prosessert i et Kalmanfilter utarbeidet i denne oppgaven. Kalmanfilteret utfører sensorfusjonering, og returnerer estimater av navigasjonsvariabler og sensorskjevheter. Videre i oppgaven har det blitt utarbeidet en tilstandsmaskin som genererer flybaner og justerer kontrollerparametere, i tillegg til at to navigasjonsmetoder er definert og sammenlignet.

Det kamerabaserte navigasjonssystemet, tilstandsestimatoren, en av navigasjonsmetodene og tilstandsmaskinen er implementert på et UML som softwaremoduler. Resultater fra et titalls fysiske tester der softwaremodulene kjører i sanntid på luftfartøyet, viser med høy robusthet og etterprøvbarhet at UML kan lokalisere og lande presist på en statisk landingsplattform. Tiden strakk dessverre ikke til for å få gjennomført de gjenstående testene på autonom landing på farkost i bevegelse. Imidlertid gir resultater fra simuleringer, og testresultater fra tilstandsestimatoren gjennomført på farkost i bevegelse, en sterk indikasjon på at systemet skal kunne gjennomføre autonom landing på farkost i bevegelse. 
%\afterpage{\null\newpage}